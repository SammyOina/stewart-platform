\chapter{Methodology}
\section{Outline}
This section looks into the mechanical designs for the Stewart platform, considerations for the force sensor and the velocity measurement within the wind tunnel. It will finally look into the design of a human machine interface that will be used to control the Stewart platform,  obtain the measurements and display results.

\section{Design of Stewart Platform}
This section deals with: design considerations, position, velocity and acceleration analysis of the Stewart platform.
\subsection{Design Considerations}
\begin{enumerate}
\item When the controllable axes are active, the platform must be controlled in six degrees of motion.
\item When the controllable members are stationary, the
platform must have a corresponding fixed position.
\item The design parameters for consideration are: mobile platform radius, height of the platform and angle between adjacent joints of mobile platform and base plate.
\item The mechanism should be lightweight.
\item Velocity and position control of the mechanism should be easy to achieve.
\end{enumerate}
\subsection{Kinematic Analysis - Rotational matrix}
Euler angles are utilized to obtain rotational matrix for the moving platform of the Stewart platform mechanism. The rotational matrix is represented as follows \cite{csumnu2017simulation}:
\begin{equation}
 R_{P}^B = R_{Z}(\gamma)*R_{Y}(\beta)*R_{X}(\alpha)
\end{equation}
Where $\gamma, \beta, \alpha$ - angles of rotation about the z-, y- and x-axis respectively.

In matrix form:
\[ R_{P}^B =
 \begin{bmatrix}
 c\beta c\gamma & s\alpha s\beta c\gamma - c\alpha s\gamma & c\alpha s\beta c\gamma + s\alpha s\gamma\\
 c\beta s\gamma & s\alpha s\beta s\gamma + c\alpha c\gamma & c\alpha s\beta s\gamma - s\alpha c\gamma\\
 -s\beta & s\alpha c\beta & c\alpha c\beta  
 \end{bmatrix}
\]
Generalized coordinate position and velocity vector of the moving platform is represented below \cite{csumnu2017simulation}:
\[
q=
\begin{bmatrix}
tx & ty & tz & \alpha & \beta & \gamma
\end{bmatrix}^\top
\]
\[
\dot{q}=
\begin{bmatrix}
\dot{tx} & \dot{ty} & \dot{tz} & \dot{\alpha} & \dot{\beta} & \dot{\gamma}
\end{bmatrix}^\top
\]
Transformation of angular velocity of moving platform to the base frame can be done using Euler angles \cite{csumnu2017simulation}:
\[
\omega =
\begin{bmatrix}
1 & 0 & s\beta \\
0 & c\alpha & -s\alpha c\beta \\
0 & s\alpha & c\alpha c\beta 
\end{bmatrix}
\begin{bmatrix}
\dot{\alpha} \\
\dot{\beta}\\
\dot{\gamma}
\end{bmatrix}
\]
Acceleration of the moving platform is obtained by differentiating the velocity of the moving platform with respect to time:
\[
\dot{\omega} =
\begin{bmatrix}
1 & 0 & s\beta \\
0 & c\alpha & -s\alpha c\beta \\
0 & s\alpha & c\alpha c\alpha c\beta 
\end{bmatrix}
\begin{bmatrix}
\ddot{\alpha} \\
\ddot{\beta}\\
\ddot{\gamma}
\end{bmatrix}
+
\begin{bmatrix}
0 & 0 & \dot{\beta}c\beta \\
0 & \dot{\alpha} s\alpha & -\dot{\alpha} c\alpha c\beta + s\alpha \dot{\beta} s\beta \\
0 & \dot{\alpha}c\alpha & -\dot{\alpha}s\alpha c\beta - c\alpha \dot{\beta}s\beta
\end{bmatrix}
\begin{bmatrix}
\dot{\alpha} \\
\dot{\beta}\\
\dot{\gamma}
\end{bmatrix}
\]
\subsection{Inverse Kinematic Analysis}
Inverse kinematic analysis is used to determine the length of the legs according to planned trajectories of the moving platform position. In order to consider the length of the leg of a Stewart platform, closed-loop of one leg is used as shown in figure 3.1. Using this closed-form representation, the leg
vector with respect to base platform can be obtained as follows \cite{csumnu2017simulation}:
\begin{equation}
\label{eqn}
L_{i} = q_{i}^{B} - b_{i}
\end{equation}
The position vector of ith upper junction point with respect to the base frame is given by the following;
\begin{equation}
\label{eqn}
q_{i}^{B} = t + R_{p}^{B} * q_{i}^{p}
\end{equation}
\begin{center}
	\begin{figure}[!h]
	\centering
	\includegraphics[width=0.6\linewidth]{Figures/Fig12}
	\caption[Closed-loop representation]{Closed-loop representation of one leg of the Stewart Platform \cite{csumnu2017simulation}}
	\end{figure}
\end{center}
Then the length of the ith leg can be acquired as follows:
\newpage
\begin{multline}
\label{eqn}
l_{i}^2 = (a_{x} * r_{p} * c_{i} + b_{x}*r_{p}*s_{i}
+ t_{x}-r_{b}*c_{i})^2 \\+ (a_{y}*r_{p}*c_{i} + b_{y}*r_{p}*s_{i} + t_{y}-r_{b}*s_{i})^2+ \\(a_{z}*r_{p}*c_{i}+b_{z}*r_{p}*s_{i}+t_{z})^2
\end{multline}
\subsection{Inverse Velocity Analysis}
Inverse Jacobian matrix can be used to perform inverse velocity analysis of the Stewart
platform. Inverse Jacobian matrix describes relation between velocity of the moving platform and the leg velocity.

Inverse Jacobian matrix for a 6-DOF Stewart Platform
\cite{csumnu2017simulation}:
\[ J^-1 =
\begin{bmatrix}
u_{1}^{T} & (R_{P}^{B}q_{1}^{B} * u_{1})^T\\
u_{2}^{T} & (R_{P}^{B}q_{2}^{B} * u_{2})^T\\
u_{3}^{T} & (R_{P}^{B}q_{3}^{B} * u_{3})^T\\
u_{4}^{T} & (R_{P}^{B}q_{4}^{B} * u_{4})^T\\
u_{5}^{T} & (R_{P}^{B}q_{5}^{B} * u_{5})^T\\
u_{6}^{T} & (R_{P}^{B}q_{6}^{B} * u_{6})^T
\end{bmatrix}
\]
\section{Base and Moving Platform}
 The Stewart platform should be stable. Thus the base will have a larger diameter than that of the moving platform. The Center of Gravity (COG) shouldn't fall outside the base during any of the platform movements.
\subsection{Stewart Plaform Configurations}
Stewart platforms can take three primary configurations i.e. 3-3 type, 3-6 type or 6-6 type.
\begin{center}
	\begin{figure}[!h]
	\centering
	\includegraphics{Figures/stewart}
	\caption[Configurations]{Type 3-3, 3-6 and 6-6 respectively
	\cite{fernandes_design_nodate}}
	\end{figure}
\end{center}
The base and the platform (for the 3-3 type), and the platform (for the 3-6 type) are triangular in shape. Theoretically and for certain dimensions, this shape gives good load carrying capability. These configurations are, however,not possible in practice since double spherical joints will be needed where two bars have to be joined at the same vertex
\cite{fernandes_design_nodate}.

Thus a 6-6 type platform is considered. The base and platform will be hexagon in shape.

\subsection{Quality Index $(\lambda)$ of Stewart Platform Design}
Quality index $(\lambda)$ varies from 0 to 1.\\
Where,\\
0 - design with singularities and\\
1 - Optimal design in which the determinant of the correspondent Jacobian matrix is maximum.

\textbf{Singularities} - uncontrollable states.

The Jacobian matrix relates the axial forces in the $i^{th}$ bar with the applied
forces and torques and is given in equation \eqref{eq:myeqn}:

\begin{equation}
J =
\begin{pmatrix}
\hat{\boldsymbol{S_{1}}} & \hat{\boldsymbol{S_{2}}} & \hat{\boldsymbol{S_{3}}} & ... & \hat{\boldsymbol{S_{n}}}
\end{pmatrix}
\label{eq:myeqn}
\end{equation}
Where n - number of bars and $ \hat{\boldsymbol{S_{i}}}$ - unit vector of the Plucker coordinates along the line of the
$i_{th}$ bar.

Quality index can be obtained as shown:
\begin{equation}
\lambda = \frac{|J|}{|J|_{m}}
\label{eq:myeqn}
\end{equation}
Where, $|J|$ is the jacobian matrix of a given Stewart platform configuration, and $ |J|_{m} $ - Jacobian Matrix of the optimal configuration.

Applying equation 3.3.1 to the general configuration of a 6-6 platform, it is
possible to obtain the Jacobian matrix and thus its determinant which can be obtained as in equation \eqref{eq:myeqn},
\begin{equation}
|J| =
\begin{pmatrix}
\frac{81 \sqrt{3} a^3 b^3 h^3 (3 \alpha \beta - 2 \alpha - 2 \beta +1)^3}{4(a^2(3 \alpha^2 - 3 \alpha + 1)+ ab(\alpha \beta - 1 )+ b^2(3 \beta^2 - 3 \beta + 1)+ 3h^2)^3}
\end{pmatrix}
\label{eq:myeqn}
\end{equation}
Where: a - diameter of moving platform
b - diameter of base
$\alpha$ - angle of attack
$ \beta $ - Side slip angle.
h - distance between the base and the moving platform.
$|J|_{m}$ is found by differentiating equation (3.3.3) with respect to h and equating to 0 resulting in \cite{fernandes_design_nodate}:
\begin{equation}
h = \sqrt{\frac{1}{3}(a^2 (3 \alpha^2 - 3 \alpha + 1)+ ab (3\alpha\beta - 1)+b^2(3 \beta^2 - 3 \beta + 1))}
\label{eq:myeqn}
\end{equation}
A base diameter of 300mm, platform diameter of 200mm and height of 190mm is considered.
\subsection{Sheet Metal Thickness and Material Selection}
Metal used to manufacture the base should be of significant thickness to provide more support to the whole structure.

Metal used to manufacture the moving platform should be lightweight and of enough thickness to enable it to withstand significant loads.

Some sheet metal materials are considered and compared.
\begin{table}[!h]
	\caption[Sheet Metal Properties]{Material Properties}
\end{table}
\begin{center}
\centering
\begin{tabular}{|l|l|l|}
\hline
\textbf{Metal} & \textbf{Density$(gcm^{-3})$} & \textbf{Yield Strength(MPa)}\\
\hline
Aluminium (6061-T6)& 2.7 & 270\\
\hline
Mild Steel & 7.85 & 250\\
\hline
Stainless Steel & 7.5 - 8.0 & 215\\
\hline
\end{tabular}
\end{center}
Aluminum 6061 is preferable for the moving platform since it is lightweight and can withstand significant loads without yielding. It also has good machinability (50 per cent). A sheet metal of 2.5mm thickness will make a suitable platform to be used in wind tunnel testing.

Both stainless steel and mild steel would make a suitable base for the Stewart platform due to their weight. They are also easy to machine (machinability for Mild steel - 78 per cent, for stainless steel - 45 - 110 per cent). Mild steel is readily available in the Kenyan market. Stainless steel is a bit more expensive.

Due to cost purposes and for the Stewart platform's overall weight (and without compromising on the platform stability), aluminium can be used to manufacture the base. A thickness of 2.5mm is selected.

Aluminium, unlike mild steel, would not require post-processing methods (e.g. powder coating) which would add on the cost.


\section{Design of the Force Sensor}
The force sensor module will be used to measure the force and moments from the aerodynamic loads applied on model being tested in the wind tunnel. The forces to be measured are the drag, lift and thrust as well as associated moments. For this subsystem two possible conceptual designs are to be considered:
\begin{itemize}
\item External force sensor
\item Stewart Platform as a force sensor
\end{itemize}
\subsection{External Force Sensor}
In this case it would require at least 3 orthogonally positioned load cells measuring each force component. Each load cell would be mechanically linked to the model such that forces experienced on each axis are measured by each load cell. 
\begin{center}
	\begin{figure}[!h]
		\centering
		\includegraphics{Figures/modBal}
		\caption[Diagram of a force balance]{Diagram of a force balance \cite{post_force_2010}}
	\end{figure}
\end{center}
This configuration is however bulky by requiring an additional external system for force measurements in addition to the stewart platform for positioning the model. This is however complemented by the simplicity in calibration of the load cells and does not require a complex force transformation matrix and other issues with force amplification created by the use of an integrated system.
\subsection{Stewart Platform as a force sensor}
In this configuration the stewart platform legs are used as force sensors by attaching strain gauges in the legs of platform. Similar work has been done by \cite{ferreira2015design} without the use of actuators as is proposed in this project. Using the stewart platform as a force sensor requires the actuators to be locked with zero degrees of freedom.

Four strain gauges are required for each leg for a full wheatstone bridge configuration. 
\begin{center}
	\begin{figure}[!h]
		\centering
		\includegraphics{Figures/loadConf}
		\caption[Strain Gauge Configuration]{Strain Gauge Configuration \cite{noauthor_measuring_nodate}}
	\end{figure}
\end{center}
In this case as shown in the figure the load cells are able to measure the axial strain on each leg. R1 and R3 are active strain gauges measuring the compressive Poisson effect (–νe). R2 and R4 are active strain gages measuring the tensile strain (+e). The output generated from the wheatstone bridge is then amplified and read to determine the strain on each leg.

\paragraph{Force measurement Circuit}
The output excitation of the wheatstone brisge needs to be amplified as it results in low outputs. An analog to digital converter is also reuiqred fot the analog to digital conveter. Some considerable options are the hx711 or the AD7193 converters which may be used as digital to analog converters. The AD7193 is designed for high precision and has a delta sigma filter to remove noise from measurements. 

The connection of the AD7193 to the microcontroller will be via Serial Peripheral Interface (SPI). the configuration is as shown in the figure:
\begin{center}
\begin{figure}
\centering
\includegraphics[width=0.55\linewidth]{Figures/SPI}
\caption[SPI configuration]{SPI configuration}
\end{figure}
\end{center}
In this configuration the when the chip select of an amplifier is set to low, the microcontroller is able to obtain data from that strain gauge. This configuration allows for a four wire interface to connect to six strain gauge sensors and obtain measurements.

\paragraph{Force transformation matrix} 
In such a case the forces experienced at the top of the platform are distributed between the 6 legs and as result, a force transformation matrix is required to resolve the forces apllied on each axis as measured by each load cell on each leg. 

If the platform is acted upon by an external wrench {$\vec{F}_e, \vec{M}_e$}, for static equilibrium of the body, the external wrench is statically balanced by the six leg forces of the stewart platform. Representing the unit vector $\hat{I}_i$ along the i-th leg with respect to B, the leg force is given  by $\hat{I}_if_i$. Considering the force equilibrium of the platform along  three mutually perpendicular directions in B(XYZ), the following force equations can be obtained as in \cite{dwarakanath_design_2001}:

$(F_e)_x = f_1I_{1x} + f_2I_{2x} + f_3I_{3x} + f_4I_{4x} + f_5I_{5x} + f_6I_{6x}$

$(F_e)_y = f_1I_{1y} + f_2I_{2y} + f_3I_{3y} + f_4I_{4y} + f_5I_{5y} + f_6I_{6y}$

$(F_e)_z = f_1I_{1z} + f_2I_{2z} + f_3I_{3z} + f_4I_{4z} + f_5I_{5z} + f_6I_{6z}$

where $(F_e)_x$, $(F_e)_y$ and $(F_e)_z$ are the external forces on the platform along three mutually perpendicular directions x, y and z of the frame B, respectively.

The moment due to the forces $\hat{I}_if_i$ about the origin of B is $(\vec{b}_i x \hat{I}_i)f_i$. Considering the moment equilibrium about x, y and z axes of B, the following moment equations can be obtained as in \cite{dwarakanath_design_2001}:

$(M_e)_x = f_1(\vec{b}_1 x \hat{I}_1)_x + f_2(\vec{b}_2 x \hat{I}_2)_x + f_3(\vec{b}_3 x \hat{I}_3)_x + f_4(\vec{b}_4 x \hat{I}_4)_x + f_5(\vec{b}_5 x \hat{I}_5)_x + f_6(\vec{b}_6 x \hat{I}_6)_x$

}$(M_e)_y = f_1(\vec{b}_1 x \hat{I}_1)_y + f_2(\vec{b}_2 x \hat{I}_2)_y + f_3(\vec{b}_3 x \hat{I}_3)_y + f_4(\vec{b}_4 x \hat{I}_4)_y + f_5(\vec{b}_5 x \hat{I}_5)_y + f_6(\vec{b}_6 x \hat{I}_6)_y$

$(M_e)_z = f_1(\vec{b}_1 x \hat{I}_1)_z + f_2(\vec{b}_2 x \hat{I}_2)_z + f_3(\vec{b}_3 x \hat{I}_3)_z + f_4(\vec{b}_4 x \hat{I}_4)_z + f_5(\vec{b}_5 x \hat{I}_5)_z + f_6(\vec{b}_6 x \hat{I}_6)_z$

where $(M_e)_x$, $(M_e)_y$ and $(M_e)_z$ are the external moments on the platform  about the three coordinate axes of B. Combining the equations the relationship between the external wrench and the forces experienced by the legs can be expressed as follows:
$$
\begin{Bmatrix}
\vec{F}_e \\
\vec{M}_e \\
\end{Bmatrix} = [H]\{F\}
$$

\section{Velocity Measurement}
An important part in wind tunnel measurements is the measure of pressure at specific points in the wind tunnel and computing the corresponding air speed. This is achieved by the use of a pitot probe. 
\begin{center}
\begin{figure}
\centering
\includegraphics[width=0.6\linewidth]{Figures/pitot}
\caption[Pitot-static tube]{Pitot-static tube \cite{noauthor_wind_nodate}}
\end{figure}
\end{center}
The equation relates the speed of the fluid at a point to both the mass density of the fluid and the pressures at the same point in the flow field. For steady flow of an incompressible fluid for which viscosity can be neglected, the fundamental equation has the form:

$$ v = \sqrt{\frac{2(P_{0} - P)}{\rho}}$$

Where V is the speed of the fluid, P0 is the total, also called the stagnation, pressure at that point of measurement, and p is the static pressure at the same point.

Three pitot probes are to be used in the wind tunnel, these are in the test section, intake and dissuser sections.

\section{Design of Human Machine Interface}
The purpose of the interface is to enable control of the platform position as well as to obtain measured data from the strain gauges and pitot tubes. 
The general layout is as shown in the figure below:
\begin{center}
\begin{figure}
\centering
\includegraphics{Figures/interface}
\caption[Human Machine Interface]{Human Machine Interface}
\end{figure}
\end{center}

The primary interface between the microcontroller is via serial interface by use of USB. This is used due to wide availability and integration on many personal computers. Serial communication allows for bidirectional transfer of information thus allowing for both control input and output of measured values. The interface is to enable the abstraction of data aquisition and actuator control to a simple interaction with buttons and other visual interfaces available in a computer program.

The decoupling of the microcontroller and the program to be hosted on the desktop allows for advanced data processing such as application of filters and resolving measurements into useable information. It also allows for the use of more powerful processor and high level programmming languages to more effectively perform complex calculations without taking a toll the sensor samping rate that would occur with onboard processing in the microcontroller.
\begin{center}
	\begin{figure}[!h]
	\centering
	\includegraphics[width=0.7\linewidth]{Figures/Fig14}
	\caption[Control Algorithm]{Control Algorithm for the Stewart Platform with Force Measurement}
	\end{figure}
\end{center}