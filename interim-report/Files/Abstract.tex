\paragraph{Nomenclature}
\begin{itemize}
\item $\vec{b}_i$ - position of ith connection point at base with respect to B
\item B - coordinate system defined at center of base plate
\item DAQA - Data Acquisition System
\item DOF - Degrees of Freedom
\item ELB - Engineering Laboratory Building
\item $\vec{F}_e$ - external force applied to platform
\item $f_i$ - force exterted on ith leg due $F_e, M_e$
\item $F$ - vector of leg forces
\item $H$ - transformation matrix which related measured forces and applied forces
\item $\hat{I}_i$ - Unit vector along the ith leg 
\item JKUAT - Jomo Kenyatta University of Agriculture and Technology
\item $\vec{M}_e$ - external momment applied to platform
\item NASA - National Space Agency
\item SPI - Serial Peripheral Interface
\end{itemize}
\pagebreak
\addcontentsline{toc}{section}{Abstract}

\section*{Abstract}
\label{sec:Abstract}
Obtaining and simulating the aerodynamic performance of items in a wind tunnel is a
significant and important part in the development of vehicles, aircraft and other machines
that require aerodynamic performance evaluation. Due to the complex maneuvers that may require simulation, there is a need for dynamic positioning of the model of the object in the wind tunnel. As a result, the proposal for a Stewart platform to replicate these complex maneuvers during wind tunnel tests as well as to position the model to obtain the required data.

This project will look into the modeling, simulation and development of a Stewart
platform based force balance for a low speed wind tunnel. The project will utilize
MATLAB/Simulink for modeling and simulation as well as Autodesk Inventor for the mechanical
design. A robust control system will also be developed for the Stewart platform.
Finally, models will be developed and tested in a wind tunnel to evaluate the performance
of the platform. The force balance and platform should be able to position the test
item and measure forces as well as calculate the aerodynamic coefficients using a bespoke computer program.




