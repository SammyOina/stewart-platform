% !TeX spellcheck = en_GB
\addcontentsline{toc}{section}{Abstract}

\section*{Abstract}
\label{sec:Abstract}
Aerodynamic analysis of the performance of models within a wind tunnel is an important step in the design and development of vehicles and aircraft. Wind tunnel testing provides a way of verifying data collected from simulations.
There is, however, need to orient models during testing, which is accomplished in this project by the use of a Stewart platform.

The Stewart platform force balance was designed to orient models with six degrees of freedom based on inverse kinematics
as well as to measure 6 components of forces and moments based on strain gauge sensors. 
A human machine interface dashboard was consequently created to allow control of the platform orientation and display of measurement results. Three dimensional models of the platform were created using Autodesk Inventor to show the configuration allowing for six degrees based on the inverse kinematic equations. 
Additonally, pitot tubes were setup in different locations in the wind tunnel and were used measure air velocity. Further,
a smoke distribution assembly was incorporated in the wind tunnel to aid in visualizing the interaction
of test models with the airflow. A printed circuit board to be used was also designed using Easy EDA and was fabricated. 


The force balance ws tested in the wind tunnel with a model aerofoil to verify its operation. Durng the test,
the platform orientation could be controlled via the dashboard, measurements were obtained and use of smoke 
streamlines improved visualization of model interaction with the airflow.