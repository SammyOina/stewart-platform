\section{Methodology}
\subsection{Outline}
This section will look into:
\begin{itemize}
	\item The instrumentation of the Stewart platform legs in order to come up with a force balance.
	\item Electromechanical modifications to improve the platform functionality.
	\item The inverse kinamatics used to come up with servo angles for Stewart platform control.
	\item The printed circuit board.
	\item The development of a Human-machine
	 Interface for measurement readings and control of the Stewart platform load balance.
	\item Pitot tubes setup in the wind tunnel and pitot tube programming.
	\item The integeration of a smoke visualization system into the wind tunnel.
	\item Computational Fluid Dynamics to determine the effects of the strut to the flow and to obtained 
		force values.
\end{itemize}

\subsection{Instrumentation of the Stewart Platform}
A 3D model of the Stewart plaform was done using Autodesk Inventor to aid in mechanical eveluations.
Finite element analysis was done using the operating conditions in table \ref{tableop}.

\begin{center}
	\begin{table}[H]
	\caption{Operating Conditions}
	\centering
	\label{tableop}
	\end{table}
	\begin{tabular}{|l|l|}
	\hline
	\textbf{Load Type} & \textbf{Force}\\
	\hline
	Magnitude & 50.0N\\
	\hline
	Vector X & 7.794N\\
	\hline
	Vector Y & -48.206N\\
	\hline
	Vector Z & 10.745N\\
	\hline
	\end{tabular}
	\end{center}
A maximum load of 50N was selected for the Finite Element Analysis.
This could be more or less since models vary in size. The z-component of the force was considered.
From FEA done on the Stewart platform, 
strain values were expected along the Stewart platform legs as shown in figure \ref{eq}.
\begin{center}
	\begin{figure}[H]
	\centering
	\includegraphics[width=0.6\linewidth]{Figures/Equivalent}
	\caption[Equivalent strain]{Strain induced on each of the legs}
	\label{eq}
	\end{figure}
\end{center}
In order to detect and convert this strains into electric signals, strain gauges in Whetastone Bridge configuration
had to be attached to the Stewart platform legs. An adhesive was required for the operation with care being
taken in first cleaning the metal surfaces before attachment. The strain gauges are delicate. Figure \ref{bon}
shows straing gauge bondin using raisin.
\begin{center}
	\begin{figure}[H]
	\centering
	\includegraphics[width=0.5\linewidth]{Figures/Strain gauge bonding.JPG}
	\caption[Bonding]{Starin gauge bonding}
	\label{bon}
	\end{figure}
\end{center}
The strain gauge wires were then soldered and covered using heat shrinkable tubing as shown in figure 
\ref{shrink}.
\begin{center}
	\begin{figure}[H]
	\centering
	\includegraphics[width=0.5\linewidth]{Figures/shrinkable1.JPG}
	\caption[Full assembly]{Strain gauge bonded to Stewart platform legs}
	\label{shrink}
	\end{figure}
\end{center}
\subsection{Electromechanical}
\subsubsection{Electromechanical Design}
Two methods of actuation are popular with Stewart platforms:
\begin{itemize}
\item Linear actuators
\item Use of motors
\end{itemize}
Whereas linear actuators are relatively easy to control, they are very expensive. Thus the choice of motors.
\subsubsection{Motor Selection}
One parameter was considered in our motor selection i.e. torque required. From the Autodesk Inventor 
design environment, the mass of the moving platform and mechanical couplings was 2.6kg. Each motor is 
supposed to move a minimum mass of 0.177kg. Using a servo horn of distance 45mm centre-to-centre, and 
Distance = $ \pi $ D, then:
\begin{ceqn}
\begin{align}
	Torque = (0.1777 \times 9.81)\times \frac{141.37}{1000} = 0.246 N\cdot m
\end{align}
\end{ceqn}
Motors of minimum torque of 0.2 N $\cdot m$ were considered. But for control considerations, only stepper 
motors and servo motors were shortlisted. Servo motor was considered over Stepper motor because they are cheaper. 
Also, we are only interested in 0 - 180 degrees positions.
The Servo motor \textbf{TowerPro SG5010} was found suitable for this project. The specifications provided by the manufacturer are as follows:

\begin{table}[!h]
\caption[Motor Specifications]{TowerPro SG5010 Specifications}
\end{table}
\begin{center}
\begin{tabular}{|l|l|}
\hline
\textbf{Model}& TowerPro SG5010\\
\hline
\textbf{Operating Voltage} & 4.8V - 6.6V\\
\hline
\textbf{Operating Speed @ 4.8V} & 0.20sec/60$^{\circ}$\\
\hline
\textbf{Operating Speed @ 6.6V}& 0.16sec/60$^{\circ}$\\
\hline
\textbf{Stall torque @ 4.8V} & 5.5 kg-cm or 0.54 N $\cdot$ m\\
\hline
\textbf{Stall torque @ 6.6V} & 6.5 kg-cm or 0.64 N $\cdot$ m\\
\hline
\end{tabular}
\end{center}
\subsubsection{Electromechanical Modification}
The Servo motor \textbf{TowerPro SG5010} used initially had plastic gears. This proved not to be capable of carrying the
platform and any load attached to it. Other servo motors (three) had been broken when we started working with them. The
servo motors were therefore replaced by the same servo motors but that have metal gears. The price was the same as plastic-generated
servo motors i.e. \textbf{KES 1,100.}

\subsection{Inverse Kinematics for Platform Control}

\subsection{Electrical}
This section looks at the design of the circuits used to power and get readings from the sensors and control the actuators.

\subsubsection*{Microcontroller}
The main requirement for this was the need for wireless communication via WIFI and large number of I/O 
interface at cost effective rate. This brought down the options to the ESP32 and 
Raspberry Pi RP2040. Both are dual core devices which are feature rich however the raspberry rp2040 only
 has 28 I/O pins while the ESP32 has 34 pins and due to this the ESP32 was chosen. 
 \begin{center}
	\begin{figure}[H]
	\centering
	\includegraphics[width=0.7\linewidth]{Figures/mcu}
	\caption[Microcontroller]{Microcontroller}
	\end{figure}
\end{center}
\subsubsection*{Power Supply}
The power supply chosen is an AC/DC power converter since the highest voltage required is 12 volts. This is also used as the highest power supply to the PCB.

Both 5 and 3.3 volts are required in the PCB for other peripherals. To this end, a buck converter and linear converter are used for 5 ad 3.3 volts respectively.

\begin{center}
	\begin{figure}[H]
	\centering
	\includegraphics{Figures/buck}
	\caption[Buck converter]{Buck Converter 12V to 5V}
	\end{figure}
\end{center}

\begin{center}
	\begin{figure}[H]
	\centering
	\includegraphics{Figures/5233}
	\caption[Linear Voltage Converter]{Linear level converter 5V to 3.3v}
	\end{figure}
\end{center}
The linear converter dissipates the energy loss though heat while buck converter is through switching.
\subsubsection*{Servo Control}
Since the microcontroller used (ESP32) is based on 3.3V logic. There is a need for 5V logic to control the servos and thus a bidirectional logic level converter is utilized which leverages an internal drain N-channel MOSFET. 
\begin{center}
	\begin{figure}[H]
	\centering
	\includegraphics{Figures/logik}
	\caption[Bidirectional Logic Level converter]{Bidirectional Logic Level converter}
	\label{fig:logik}
	\end{figure}
\end{center}
The configuration designed is shown in figure \ref{fig:logik}.
\subsubsection{Printed Circuit Board}
The schematics from the electrical designs as seen in the appendix were routed to generate a PCB ready for manufacture. This is shown in the 3D render below:
\begin{center}
	\begin{figure}[H]
	\centering
	\includegraphics[width=0.45\linewidth]{Figures/pcb}
	\caption[Printed Circuit Board view]{Printed Circuit Board view}
	\label{fig:pcb3d}
	\end{figure}
\end{center}
The PCB is meant to be compact as shown in the figure \ref{fig:pcb3d}. It is also integrates most of the components to minimize use of wires to make connections.
\subsubsection{Manufactured PCB}
The PCB was manufactured and delivered and is shown in figure \ref{final pcb}.
\begin{center}
	\begin{figure}[H]
	\centering
	\includegraphics[width=0.45\linewidth]{Figures/pcb 1.JPG}
	\caption[Final PCB]{Manufactured PCB pictured alongside an RS-485}
	\label{final pcb}
	\end{figure}
\end{center}

\subsection{Human Machine Interface Design and Development}
\subsection{Velocity Measurement}
Velocity measurement is to be acheieved in the wind tunnel by use of pitot tubes. Three pitot tubes are
required - one at the test section and is attached on the strut, one at the inlet and the other at the diffuser as shown
in figure \ref{pitot placement}.
\begin{center}
	\begin{figure}[H]
	\centering
	\includegraphics[width=0.45\linewidth]{Figures/wt and pitot.JPG}
	\caption[Pitot tubes setup]{Pitot tubes setup in the wind tunnel}
	\label{pitot placement}
	\end{figure}
\end{center}
From left to right in figure \ref{pitot placement} above is the inlet, the test section (above the Stewart platform) and the diffuser.
The pitot tube in the inlet section is placed exactly where the test section begins and the pitot tube at the
diffuser section is placed exactly where the test section ends. The pitot probes are set at height that as close to the 
middle (in terms of height) of flow as possible. Figure \ref{pitot height} shows shows the height of the pitot tube in the wind tunnel.
\begin{center}
	\begin{figure}[H]
	\centering
	\includegraphics[width=0.25\linewidth]{Figures/Pitot in diffuser.jpg}
	\caption[Pitot tubes setup in diffuser]{Pitot tube in the diffuser section of the wind tunnel}
	\label{pitot height}
	\end{figure}
\end{center}
Holes for inserting the pitot tubes were drilled using a hand drill and a 7mm diameter drill bit. Pitot tube has a diameter of 6mm.
\subsection{Smoke Visualization System}
To implement streamline smoke lines in the wind tunnel, a smoke rake, which is an aerodynamically shaped
body (typically elliptical) featuring a row of tubes through which the smoke exits was designed. 
The CAD model was deigned using Autodesk Inventor and the rake was prepared for 3D printing 
at iPIC in JKUAT. The rake design is shown below.
\begin{center}
	\begin{figure}[H]
	\centering
	\includegraphics[width=0.55\linewidth]{Figures/rake design.JPG}
	\caption[Rake design]{Rake design used to produces smoke streamlines}
	\label{pitot height}
	\end{figure}
\end{center}
The printed rake is shown in figure \ref{rake printed}.
\begin{center}
	\begin{figure}[H]
	\centering
	\includegraphics[width=0.25\linewidth]{Figures/smoke emitter.jpg}
	\caption[3D Printed rake]{3D Printed rake}
	\label{rake printed}
	\end{figure}
\end{center}
To implement smoke visualisation in the wind tunnel, the rake is used in conjuction with:
\begin{enumerate}
	\item A fog generator
	\item A metal pipe
	\item A hose pipe and
	\item A smoke connector
\end{enumerate}
The fog, produced by heating a water-based liquid, features good dispersal properties and has a density very similar to that of air when in 
vapour form, so that the smoke lines are expected to follow the air in the test section with high fidelity and not disperse or sink too quickly. As the fog 
liquid is water-based, the resulting fog cloud is non-hazardous and thus safe to use indoors \cite{trinder2013development}.
\begin{center}
	\begin{figure}[H]
	\centering
	\includegraphics[width=0.25\linewidth]{Figures/smoke generator.jpg}
	\caption[Fog generator]{Fog generator with funnel (smoke connector)}
	\label{pitot height}
	\end{figure}
\end{center}

The metal pipe is for rigid support of the rake in the wind tunnel. From wind tunnel dimensions, a pipe of 400mm length 
was selected to place the rake exactly in the middle (vertically) of the intake section. A diameter of 3/4" (19.05mm) was chosen in order to 
ensure sufficient smoke supply and also for ease of pipe threading. A sheet plate was also included in the setup for ease of setting up the pipe 
on the wid tunnel floor. The design of the assembly is shown in figure \ref{pipe and rake}.
\begin{center}
	\begin{figure}[H]
	\centering
	\includegraphics[width=0.35\linewidth]{Figures/pipe and rake.JPG}
	\caption[Pipe and rake assembly]{Design assembly of the rake and pipe in the Wind tunnel}
	\label{pipe and rake}
	\end{figure}
\end{center}
Due to lack of drill cutters to drill a 20mm hole on the intake section floor, we had to improvise by trepanning a 20mm hole using a 7mm drill bit and hand drill.
The pipe was threaded on one end for a length of 30mm at the plumbing workshop to enable it to be fastened onto the 
wind tunnel floor using plumbing nuts (one on top and one on the bottom). Further, the pipe surface was made rough by use of a hacksaw
in order to reduce any turbulence in the flow that will result from the pipe being in the intake section. The final set up is shown in 
figure \ref{final pipe}.
\begin{center}
	\begin{figure}[H]
	\centering
	\includegraphics[width=0.35\linewidth]{Figures/final pipe.jpg}
	\caption[Pipe and rake final assembly]{Final assembly of pipe and rake in the wind tunnel}
	\label{final pipe}
	\end{figure}
\end{center}
A hose pipe is used to connect the fog generator to the matal pipe via a smoke connector. 
\subsection{Strut}
The strut will go into the test section of the Wind tunnel at the Fluids lab in JKUAT. 
The strut should be long enough to position the model under testing in the middle of the test section.
 The strut will also be subjected to the flow forces in the wind tunnel. 
 Its design therefore considered as small diameter as possible. The strut dimensions of diameter 12.5mm 
 and a length 400mm are practical, based on the Wind Tunnel set-up (Appendix) \cite{caleb}.
\begin{center}
	\begin{figure}[H]
	\centering
	\includegraphics[width=0.75\linewidth]{Figures/Strut}
	\caption[Strut]{Strut}
	\end{figure}
\end{center}
\subsubsection{Computational Fluid Dynamics of the Strut}
\subsubsection{Finite Element Analysis}
A Finite Element Analysis was done on the Stewart platform design to determine its performance under given loads. The platform is going to be subjected to a force during wind tunnel testing. A maximum load of 50N is selected for the Finite Element Analysis. This could be more or less since models vary in size. The z-component of the force is considered. 
The mesh generated is shown in the figure \ref{fig:feamesh}.
\begin{center}
	\begin{figure}[H]
	\centering
	\includegraphics[width=0.75\linewidth]{Figures/FEA}
	\caption[Stewart platform mesh]{Stewart platform mesh}
	\label{fig:feamesh}
	\end{figure}
\end{center}
The rods/legs of the Stewart platform are subjected to axial forces which cause slight strains which can be used to obtain force and moment measurements during wind tunnel testing.
\clearpage
\begin{center}
\begin{table}[H]
\caption{Operating Conditions}
\centering
\end{table}
\begin{tabular}{|l|l|}
\hline
\textbf{Load Type} & \textbf{Force}\\
\hline
Magnitude & 50.0N\\
\hline
Vector X & 7.794N\\
\hline
Vector Y & -48.206N\\
\hline
Vector Z & 10.745N\\
\hline
\end{tabular}
\end{center}

\begin{center}
\begin{table}[!h]
\caption[FEA Setup]{Other FEA setup parameters}
\centering
\end{table}
\begin{tabular}{|l|l|}
\hline
Constraint & Fixed\\
\hline
Materials & Aluminium 6061\\
 & Stainless steel\\
\hline
Contacts & All bonded\\
\hline
Mesh& Avg element size: 0.1mm\\
& Minimum element size: 0.2mm\\
& Grading Factor: 1.50\\
& Maximum turn angle: 60 deg\\
\hline
\end{tabular}
\end{center}
\subsubsection{Inverse Kinematics of a Stewart Platform}
This is the calculation of each leg length based on the desired position of the Stewart platform. 
For the Stewart platform, the translation $^{p}T_{b}$ from the base origin to the platform origin can be described with a single vector T = $(t_{x} t_{y} t_{z})^{T} $. The rotation of the platform can be denoted as $^{p}R_{b}$. Thus the following relationship for the frame of reference can be stated \cite{Eisele_2019}:
	\begin{ceqn}
	\begin{align}
	^{p}T_{b} = T
	\end{align}
	\end{ceqn}
	\begin{ceqn}
	\begin{align}
	^{p}R_{b} = R
	\end{align}
	\end{ceqn}
	\begin{ceqn}
	\begin{align}
	^{b}T_{p} =(^{p}T_{b})^{-1} =T^{-1}
	\end{align}
	\end{ceqn}
	\begin{ceqn}
	\begin{align}
	^{b}R_{p} = (^{p}R_{b})^{-1}
	\end{align}
	\end{ceqn} 
\begin{center}
	\begin{figure}[!h]
	\centering
	\includegraphics[width=0.4\linewidth]{Figures/servo2}
	\caption[Leg length]{Leg Length \cite{Eisele_2019}}
	\end{figure}
\end{center}
The leg length $l_{k}$ can therefore be found with:
\begin{ceqn}
\begin{align}
	l_{k} = P_{k} - B_{k} = T + R \times p \times R^{-1} - b_{k}
\end{align}
\end{ceqn}
Where,\\
$P_{k}$ - leg attachment on platform \\
$B_{k}$- leg attachment on base 
\subsubsection{Inverse Kinematics Using Rotational Servo Motors}
For a rod of fixed length d between servo horn anchor $H_{k}$ and platform anchor $P_{k}$.  $H_{k}$ has a distance h between the original base anchor and servo shaft $B_{k}$. The vector h is perpendicular to the servo shaft and is rotated by angle $\alpha_{k}$ when lifted from the horizontal line.
\begin{center}
	\begin{figure}[!h]
	\centering
	\includegraphics[width=0.4\linewidth]{Figures/servo}
	\caption[Servo angle]{Servo angle \cite{Eisele_2019}}
	\end{figure}
\end{center}
Further, looking from the top onto the base, each servo can be rotated by the angle $\beta_{k}$ in addition to its position $B_{k}$. The servo shaft $s_{k}$ moves in the x-y plane and is orthogonal to the vector h.
\begin{center}
	\begin{figure}[!h]
	\centering
	\includegraphics[width=0.4\linewidth]{Figures/servo1}
	\caption[Planar view]{Planar view \cite{Eisele_2019}}
	\end{figure}
\end{center}
The anchor $H_{k}$ can be calculated as we rotate around the z-axis with angle $\beta_{k}$ and  by $-\alpha_{k}$ around the y-axis to lift the servo horn which lies in the local x-axis \cite{Eisele_2019}.
\begin{ceqn}
\begin{align}
	H_{k} = B_{k} + R_{z}(\beta_{k}) R_{y}(-\alpha_{k})\begin{pmatrix}
	|h|\\ 0 \\ 0
	\end{pmatrix}
	\label{eq:myeqn2}
\end{align}
\end{ceqn}
\begin{ceqn}
 \begin{align}
	= B_{k} + |h|\begin{pmatrix}
	cos(\alpha_{k})cos(\beta_{k})\\
	 cos(\alpha_{k})sin(\beta_{k})\\
	  sin(\alpha_{k})
	\end{pmatrix}
	\label{eq:myeqn2}
\end{align}
\end{ceqn}
Carrying out the same procedure but with the servo arm when it is on the opposite side and we have:
\begin{ceqn}
\begin{align}
	H_{k} = B_{k} + R_{z}(\beta_{k}) R_{y}(\pi -\alpha_{k})\begin{pmatrix}
	-|h|\\ 0 \\ 0
	\end{pmatrix}
	\label{eq:myeqn2}
\end{align}
\end{ceqn}
\begin{ceqn}
	\begin{align}
	= B_{k} + |h|\begin{pmatrix}
	cos(\alpha_{k})cos(\beta_{k})\\
	 cos(\alpha_{k})sin(\beta_{k})\\
	  sin(\alpha_{k})
	\end{pmatrix}
	\label{eq:myeqn2}
\end{align}
\end{ceqn}
Squaring the lengths of h, d, $l_{k}$ we get the relationship:
\begin{ceqn}
	\begin{align}
		h^2 = (H_{k}-B_{k})^{T}(H_{k}-B_{k})
	\end{align}
\end{ceqn}
\begin{ceqn}
	\begin{align}
		d^2 = (P_{k}-H_{k})^{T}(P_{k}-H_{k})
	\end{align}
\end{ceqn}
\begin{ceqn}
	\begin{align}
		l_{k}^2 = (P_{k}-B_{k})^{T}(P_{k}-B_{k})
	\end{align}
\end{ceqn}
From the above relationships, further derivation and trigonometric substitutions are done to obtain the inverse kinematics used to calculate each servo angle. This is given as \cite{Eisele_2019}:
\begin{ceqn}
\begin{align}
	\alpha_{k} = sin^{-1}(\frac{g_{k}}{\sqrt{e_{k}^2+f_{k}^2}})-arctan2(f_{k}, e_{k})
\end{align}
\end{ceqn}
Where,
$$e_{k} = 2hl_{k}^{(z)} $$
$$e_{k} = 2h(cos(\beta_{k})I_{k}^{(x)}+sin({\beta_{k}})I_{k}^(y))$$
$$g_{k} = l_{k}^2 - (d^2 - h^2)   $$

\subsection{Electromechanical Design}
Two methods of actuation are popular with Stewart platforms:
\begin{itemize}
\item Linear actuators
\item Use of motors
\end{itemize}
Whereas linear actuators are relatively easy to control, they are very expensive. Thus the choice of motors.

\subsubsection{Motor Selection}
One parameter was considered in our motor selection i.e. torque required. From the Autodesk Inventor design environment, the mass of the moving platform and mechanical couplings was 2.6kg. Each motor is supposed to move a minimum mass of 0.177kg. Using a servo horn of distance 45mm centre-to-centre, and Distance = $ \pi $ D, then:
\begin{ceqn}
\begin{align}
	Torque = (0.1777 \times 9.81)\times \frac{141.37}{1000} = 0.246 N\cdot m
\end{align}
\end{ceqn}
Motors of minimum torque of 0.2 N $\cdot m$ were considered. But for control considerations, only stepper motors and servo motors were shortlisted. Servo motor was considered over Stepper motor because they are cheaper. Also, we are only interested in 0 - 180 degrees positions.

The Servo motor \textbf{TowerPro SG5010} is suitable for this project. The specifications provided by the manufacturer are as follows:

\begin{table}[!h]
\caption[Motor Specifications]{TowerPro SG5010 Specifications}
\end{table}
\begin{center}
\begin{tabular}{|l|l|}
\hline
\textbf{Model}& TowerPro SG5010\\
\hline
\textbf{Operating Voltage} & 4.8V - 6.6V\\
\hline
\textbf{Operating Speed @ 4.8V} & 0.20sec/60$^{\circ}$\\
\hline
\textbf{Operating Speed @ 6.6V}& 0.16sec/60$^{\circ}$\\
\hline
\textbf{Stall torque @ 4.8V} & 5.5 kg-cm or 0.54 N $\cdot$ m\\
\hline
\textbf{Stall torque @ 6.6V} & 6.5 kg-cm or 0.64 N $\cdot$ m\\
\hline
\end{tabular}
\end{center}
Prices range between \textbf{Ksh.700} and \textbf{Ksh.1400}

\subsubsection{Design of the Force Sensor}
The force sensor module will be used to measure the force and moments from the aerodynamic loads applied on model being tested in the wind tunnel. The forces to be measured are the drag, lift and thrust as well as associated moments. For this subsystem two possible conceptual designs are to be considered:
\begin{itemize}
\item External force sensor
\item Stewart Platform as a force sensor
\end{itemize}
\subsubsection*{External Force Sensor}
In this case it would require at least 3 orthogonally positioned load cells measuring each force component. Each load cell would be mechanically linked to the model such that forces experienced on each axis are measured by each load cell as shown in figure \ref{fig:balex}. 
\begin{center}
	\begin{figure}[H]
		\centering
		\includegraphics{Figures/modBal}
		\caption[Diagram of a force balance]{Diagram of a force balance \cite{post_force_2010}}
		\label{fig:balex}
	\end{figure}
\end{center}
This configuration is, however, bulky since it requires an additional external system for force measurements in addition to the Stewart platform for positioning the model. This is however complemented by the simplicity in calibration of the load cells and does not require a complex force transformation matrix and other issues with force amplification created by the use of an integrated system.
\subsubsection*{Stewart Platform as a force sensor}
In this configuration the Stewart platform legs are used as force sensors by attaching strain gauges on the legs of platform. Similar work has been done by \cite{ferreira2015design} without the use of actuators as is proposed in this project. Using the Stewart platform as a force sensor requires the actuators to be locked with zero degrees of freedom.

Four strain gauges are required for each leg for a full Wheatstone bridge configuration. 
\begin{center}
	\begin{figure}[H]
		\centering
		\includegraphics{Figures/loadConf}
		\caption[Strain Gauge Configuration]{Strain Gauge Configuration \cite{noauthor_measuring_nodate}}
		\label{strain}
	\end{figure}
\end{center}
In this case, as shown in figure \ref{strain}, the load cells are able to measure the axial strain on each leg. R1 and R3 are active strain gauges measuring the compressive Poisson effect (–νe). R2 and R4 are active strain gages measuring the tensile strain (+e). The output generated from the Wheatstone bridge is then amplified and read to determine the strain on each leg.
In the case of this project, a full bridge strain gauge sensor with each strain gauge orthogonally positioned allowing for elimination of noise and measurement of the strain axially on each rod is used.

%\paragraph{Force measurement Circuit}
\subsubsection*{Force measurement Circuit}

The output excitation of the Wheatstone bridge needs to be amplified as it results in low outputs. An analogue to digital converter is also required for the analogue to digital converter. Some considerable options are the HX711 or the AD7193 converters which may be used as digital to analogue converters. The AD7193 is designed for high precision and has a delta sigma filter to remove noise from measurements.  However, the AD7193 was not used due to it's high cost compared to the HX711 regardless of the advantage it offers with serial peripheral Interface communication allowing for multiple modules on the same bus.

The connection of the AD7193 to the microcontroller will be via Serial Peripheral Interface (SPI). The configuration is as shown in figure \ref{spi}:
\begin{center}
\begin{figure}[H]
\centering
\includegraphics[width=0.55\linewidth]{Figures/SPI}
\caption[SPI configuration]{SPI configuration}
\label{spi}
\end{figure}
\end{center}
In this configuration the when the chip select pin of an amplifier is set to low, the microcontroller is able to obtain data from that strain gauge. This configuration allows for a four wire interface to connect to six strain gauge sensors and obtain measurements.

The HX711 however allows for the I2C like communication via two wires (serial clock (SCL) and Serial Data (SDA)). They however do not have hardware addresses thus many pins are required for the interface, This can however be remedied by the use of a shared SCL between each of the amplifier modules reducing the overall number of pins required to interface.

\subsubsection*{Force transformation matrix} 
In such a case the forces experienced at the top of the platform are distributed between the 6 legs and as result, a force transformation matrix is required to resolve the forces applied on each axis as measured by each load cell on each leg. 

The forces experienced by the legs can be expressed as:
\begin{ceqn}
\begin{align}
	\begin{Bmatrix}
		\vec{F}_e \\
		\vec{M}_e \\
	\end{Bmatrix} = [H]\{F\}
\end{align}
\end{ceqn}

Where 
\begin{itemize}
\item $\vec{F}_e $ - external force applied to the platform
\item $\vec{M}_e$ - external moment applied to the platform
\item H - transformation matrix which relates measured forces and applied forces
\item F - axial force
\end{itemize}  
The derivation has been done and is presented in the appendix.


\subsection{Velocity Measurement}
An important part in wind tunnel measurements is the measure of pressure at specific points in the wind tunnel and computing the corresponding air speed. This is achieved by the use of a pitot probe. 
\begin{center}
\begin{figure}[H]
\centering
\includegraphics[width=0.6\linewidth]{Figures/pitot}
\caption[Pitot-static tube]{Pitot-static tube \cite{viquerat_continuous_2006}}
\end{figure}
\end{center}
For steady flow of an incompressible fluid for which viscosity can be neglected, the fundamental equation has the form\cite{viquerat_continuous_2006}:

\begin{ceqn}
	\begin{align}
	v = \sqrt{\frac{2(P_{0} - P)}{\rho}}
\end{align}
\end{ceqn}

Where V is the speed of the fluid, $P_{0}$ is the total, also called the stagnation, pressure at that point of measurement, and P is the static pressure at the same point.

Three pitot probes are to be used in the wind tunnel i.e. in the test section, intake and diffuser sections.
                                                                                                                                                                   
\begin{center}
\begin{figure}[H]
\centering
\includegraphics{Figures/modbus}
\caption[RS485 communication]{RS485 communication}
\label{fig:rs485}
\end{figure}
\end{center}
Module communicates to the microcontroller using a TTL to RS485. Each module is individually addressable further simplifying the interface as shown in figure \ref{fig:rs485}. 
The protocol allows for long distance serial communication and interface with multiple devices on the same bus.

\subsection{Control system}
This section deals with the control of the stewart platform. It involves allowing input and obtaining output form the platform force balance.
\subsubsection{Human Machine Interface}
The purpose of the interface is to enable control of the platform position as well as to obtain measured data from the strain gauges and pitot tubes. 
The general layout is as shown in the figure \ref{fig:hmi}:
\begin{center}
\begin{figure}[H]
\centering
\includegraphics{Figures/Interface}
\caption[Human Machine Interface]{Human Machine Interface}
\label{fig:hmi}
\end{figure}
\end{center}

The primary interface between the microcontroller is wireless. 
For this interface two transport methods were considered: User Datagram Protocol (UDP) and Transmission Control protocol (TCP).
UDP is a communications protocol that is primarily used to establish low-latency and loss-tolerating connections.
 TCP on the hand does not experience data loss and in our case the use of protocol buffers as a means of serialization of messages to binary saving on bandwidth. 
 TCP requires only one initial handshake between the server and client, after which messages can be transmitted continuously (streaming).
 Due to assurance of data integrity (no loss) and lower bandwidth use thus higher data transfer rates, TCP was chosen as the main protocol.
  This allows for high speed low latency communication without need for a wired connection. It also allows for bidirectional streaming of information thus allowing for both control input and output of measured values. The interface is to enable the abstraction of data acquisition and actuator control to a simple interaction with buttons and other visual interfaces available in a computer program.

The decoupling of the microcontroller and the program to be hosted on the desktop allows for advanced data processing such as application of filters and resolving measurements into usable information. It also allows for the use of more powerful processor and high level programming languages to more effectively perform complex calculations without taking a toll the sensor sampling rate that would occur with on board processing in the microcontroller.

A user interface was also required to record user input and display incoming data.
\subsubsection{Control Algorithm}
\begin{center}
	\begin{figure}[H]
	\centering
	\includegraphics[width=0.7\linewidth]{Figures/Flow}
	\caption[Control Algorithm]{Control Algorithm for the Stewart Platform with Force Measurement}
	\end{figure}
\end{center}
\subsection{Electrical Design}
This section looks at the required electrical connections to make the system functional and interface with the control unit.
\subsubsection{Printed Circuit Board Design}
This section looks at the design of the circuits used to power and get readings from the sensors and control the actuators.

\subsubsection{Microcontroller}
The main requirement for this is the need for wireless communication via WIFI and large number of I/O interface at cost effective rate. This brings down the options to the ESP32 and Raspberry Pi RP2040. Both are dual core devices which are feature rich however the raspberry rp2040 only has 28 I/O pins while the ESP32 has 34 pins and due to this the ESP32 was chosen. 
\begin{center}
	\begin{figure}[H]
	\centering
	\includegraphics[width=0.7\linewidth]{Figures/mcu}
	\caption[Microcontroller]{Microcontroller}
	\end{figure}
\end{center}
\subsubsection{Power Supply}
The power supply chosen is an AC/DC power converter since the highest voltage required is 12 volts. This is also used as the highest power supply to the PCB.

Both 5 and 3.3 volts are required in the PCB for other peripherals. To this end, a buck converter and linear converter are used for 5 ad 3.3 volts respectively.

\begin{center}
	\begin{figure}[H]
	\centering
	\includegraphics{Figures/buck}
	\caption[Buck converter]{Buck Converter 12V to 5V}
	\end{figure}
\end{center}

\begin{center}
	\begin{figure}[H]
	\centering
	\includegraphics{Figures/5233}
	\caption[Linear Voltage Converter]{Linear level converter 5V to 3.3v}
	\end{figure}
\end{center}
The linear converter dissipates the energy loss though heat while buck converter is through switching.

\subsubsection{Servo Control}
Since the microcontroller used (ESP32) is based on 3.3V logic. There is a need for 5V logic to control the servos and thus a bidirectional logic level converter is utilized which leverages an internal drain N-channel MOSFET. 
\begin{center}
	\begin{figure}[H]
	\centering
	\includegraphics{Figures/logik}
	\caption[Bidirectional Logic Level converter]{Bidirectional Logic Level converter}
	\label{fig:logik}
	\end{figure}
\end{center}
The configuration designed is shown in figure \ref{fig:logik}