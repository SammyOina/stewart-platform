\section{Introduction}
\label{sec:introduction}
Obtaining and simulating the aerodynamic performance of items in a wind tunnel is a
significant and important part in the development of vehicles, aircraft and other machines
that require aerodynamic performance evaluation. For most wind tunnel experiments where existing force balances are used,
it is only possible to alter the object's angle of attack by use of linear  actuators for example.
However, due to the complex maneuvers that
may require simulation, there is a need for dynamic positioning of the model of the object
in the wind tunnel. A six-DOF Stewart platform can be used to achieve these dynamic positions and can be used as a force balance.
\subsection{Background}
A Stewart platform is a six degree-of-freedom mechanism that consists of a base and a platform connected with six legs. In some Stewart platforms,
each leg is connected to the ground by a two-axis joint and is provided with controllable means for extending its length
\cite{wittenburg2016stewart}. The six DOF Stewart Platform provides an elegant design for simulating flight conditions.\cite{stewart1965platform}. 

The Stewart platform can be used in conjunction with wind tunnels for wind tunnel experiments. Wind tunnels are used to measure the aerodynamic forces on airplanes, wings, cars, trucks, bridges, and buildings. This can be done by positioning models of these vehicles in the tunnel's test section. 
The Stewart platform can provide various dynamic positions to simulate complex vehicle maneuvers. Incorporating a force balance with the Stewart platform will allow direct measurement of forces and torques acting on the model that is being tested in the wind tunnel.

For this project, the force balance is intended to be built as simple and accessible as possible. Thus, the use of a six DOF Stewart Platform as a force balance. 
A Stewart Platform has already been designed and fabricated by a previous group but is not integrated with a force balance. 
Electrical load measurement devices have to be connected to the legs of the Stewart platform in order to measure and obtain the results thus making it fully operational \cite{caleb}.

\subsection{Problem Environment}
Simulation and analysis of scaled models is an important step in the development of aircraft, vehicles and other machines. Such analysis provides aerodynamic performance data that can be used to inform any modifications or improvements e.g. in aircraft and vehicles to make them more efficient and safer. One such method that is used to perform aerodynamic performance evaluation is the wind tunnel used in conjunction with sensors for data acquisition by a computer. External or internal six-component force balances are also used. Another such technology that can be used for this purpose is the Stewart platform, which can be used to simulate the 
orientation of vehicles and aircraft in the actual environment.

Whereas the force balances give very accurate results, some are expensive to build and use. Also, some objects require complex maneuver simulations to imitate the actual movements in air. There is therefore the need for dynamic positioning of objects in the wind tunnel.
\subsection{Problem Statement}
Measured parameters obtained in aerodynamic testing are critical in the development of parts. Current commercial force balances provide means to position models and obtained forces and moments during testing. 
However, these force balances are quite expensive and some do not provide an integrated solution. This project goes trough the development of a stewart platform force balance which aims to provide a cost effective solution for the low speed wind tunnel available at the Engineering fluids lab at JKUAT. 
The stewart platform force balance is meant to be able to measure 6 component aerodynamic loads as well as position the model within the wind tunnel with six degrees of freedom.
\subsection{Objectives}
\subsubsection{Main Objective}
\paragraph{} To develop a force balance and a control system for the Stewart platform. 
\subsubsection{Specific Objectives}
\begin{enumerate}
\item To evaluate the mechanical feasibility of the already fabricated Stewart Platform using CAD software and mechanical calculations.
\item To develop a force balance for the Stewart platform and obtain forces and moments measurement during model testing.
\item To obtain flow velocity measurement in the wind tunnel by use of pitot tubes.
\item To develop a Human-Machine Interface for measurement readings and control of the Stewart platform load balance.
\end{enumerate}
\subsection{Expected Outcomes}
\begin{enumerate}
\item A functional force balance for measuring the aerodynamic loads applied on a model
and for calculating aerodynamic coefficients from the measurements taken.
\item A velocity measurement system for the wind tunnel using pitot tubes.
\item A human machine interface with access to control of the Stewart platform electronics
and data acquisition from sensors.
\end{enumerate}
